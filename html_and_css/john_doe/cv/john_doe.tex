\documentclass[a4paper,10pt]{article}
\usepackage[T1]{fontenc}
\usepackage[utf8]{inputenc}
\usepackage[polish]{babel}
\usepackage{geometry}
\geometry{margin=1in}
\usepackage{titlesec}
\usepackage{xcolor}
\usepackage{hyperref}
\hypersetup{
    colorlinks=true,
    linkcolor=blue,
    urlcolor=blue
}
\titleformat{\section}{\large\bfseries}{}{0em}{}

\begin{document}

\begin{center}
    {\LARGE \textbf{John Doe}}\\
    {\large Programista Web (HTML, CSS, PHP, JS, Vue.js)}\\
    \vspace{0.5cm}
    \textbf{Email:} johndoe@example.com \hspace{1cm} \textbf{Telefon:} +48 123 456 789 \\
    \textbf{GitHub:} \url{https://github.com/johndoe} \hspace{1cm} \textbf{LinkedIn:} \url{https://linkedin.com/in/johndoe} 
\end{center}

\section{Podsumowanie zawodowe}
Programista webowy z wieloletnim doświadczeniem w tworzeniu nowoczesnych aplikacji internetowych. Specjalizuję się w technologiach frontendowych i backendowych, takich jak HTML, CSS, JavaScript, PHP oraz framework Vue.js. Aktualnie prowadzę własną firmę, świadcząc usługi programistyczne dla klientów z różnych branż.

\section{Doświadczenie zawodowe}
\textbf{Właściciel firmy IT} 
\hfill \textit{Obecnie}\\
Prowadzenie własnej działalności w zakresie tworzenia i wdrażania aplikacji internetowych. Obsługa klientów indywidualnych oraz firmowych, tworzenie dedykowanych rozwiązań webowych, optymalizacja i utrzymanie stron.

\textbf{XYZ Software} – Programista Web 
\hfill \textit{2018 – 2023}\\
- Tworzenie i rozwój aplikacji webowych w PHP i Vue.js. 
- Projektowanie responsywnych interfejsów użytkownika w HTML i CSS. 
- Integracja zewnętrznych API oraz optymalizacja wydajności aplikacji.

\textbf{ABC Tech} – Młodszy Programista 
\hfill \textit{2015 – 2018}\\
- Praca nad frontendem i backendem aplikacji webowych.
- Implementacja skryptów w JavaScript i PHP.
- Wsparcie techniczne i naprawa błędów w kodzie.

\section{Umiejętności}
- Programowanie: HTML, CSS, JavaScript, PHP, Vue.js
- Bazy danych: MySQL, PostgreSQL
- Narzędzia: Git, Webpack, Docker
- Optymalizacja i wydajność aplikacji webowych
- Tworzenie responsywnych interfejsów użytkownika

\section{Wykształcenie}
\textbf{Uniwersytet Informatyki} – Informatyka Stosowana 
\hfill \textit{2011 – 2015}\\
Specjalizacja: Programowanie aplikacji internetowych

\section{Certyfikaty i kursy}
- Certyfikat PHP Developer – 2020
- Kurs Zaawansowanego JavaScript – 2019

\section{Języki obce}
- Polski – ojczysty
- Angielski – zaawansowany (C1)

\section{Zainteresowania}
- Nowoczesne technologie webowe
- Rozwój aplikacji SPA (Single Page Applications)
- Open-source i projekty społecznościowe

\end{document}

